\documentclass[paper=letter, fontsize=11pt]{scrartcl} % Letter paper and 11pt font size

\usepackage{amstext, amsmath, amssymb}
\usepackage[T1]{fontenc} % Use 8-bit encoding that has 256 glyphs
\usepackage[english]{babel} % English language/hyphenation
\usepackage{amsmath,amsfonts,amsthm} % Math packages

\usepackage{fancyhdr} % Custom headers and footers
\pagestyle{fancyplain} % Makes all pages in the document conform to the custom headers and footers
\fancyhead{} % No page header
\fancyfoot[L]{} % Empty left footer
\fancyfoot[C]{} % Empty center footer
\fancyfoot[R]{\thepage} % Page numbering for right footer
\renewcommand{\headrulewidth}{0pt} % Remove header underlines
\renewcommand{\footrulewidth}{0pt} % Remove footer underlines
\setlength{\headheight}{13.6pt} % Customize the height of the header
\setlength\parindent{0pt} % Remove all indentation from paragraps.

%----------------------------------------------------------------------------------------
%	TITLE SECTION
%----------------------------------------------------------------------------------------

\newcommand{\horrule}[1]{\rule{\linewidth}{#1}} % Create horizontal rule command with 1 argument of height

\title{	
\normalfont \normalsize 
\textsc{San Francisco State University} \\ [25pt]
\horrule{0.5pt} \\[0.4cm] % Thin top horizontal rule
\huge MATH 301 Assignment 8 \\ % The assignment title
\horrule{2pt} \\[0.5cm] % Thick bottom horizontal rule
}

\author{Omar Sandoval}

\date{\normalsize\today}

\begin{document}

\maketitle

%----------------------------------------------------------------------------------------
%	PROBLEM #2
%----------------------------------------------------------------------------------------
\textbf{2.} Let $a$ be an integer. Prove that $a^2$ is divisible by 5 if and only if $a$ 
is divisible by 5. \\

We are given that $a \in \mathbb{Z}$ and $5$ divides $a^2$. We want to show that 5 will 
divide $a$. \\
Proof by Contradiction: Given that $a \in \mathbb{Z}$, 5 divides $a^2$, 5 does not divide
$a$.
If $a$ divides 5, $a = 5q$ and $a^2 = 25q^2 = 5(5q^2)$ is divisible by 5. \\
For converse, suppose $a^2$ is divisible by 5 and for contradiction, that $a$ is not 
divisible by 5. \\
Then, by the division theorem, $a = 5q + 1$ or $a = 5q + 2$. \\
But, $a = 5q+1 \Rightarrow a^2 = (5q+1)^2 = 25q^2 + 10q + 1 = 5(5q^2 + 2q + 1) - 1$. \\

So, in either case the remainder is non-zero, thus, by the division theorem, $a^2$ is not
divisible by 5, so we reach a contradiction. \\

%----------------------------------------------------------------------------------------
%	PROBLEM #3
%----------------------------------------------------------------------------------------
\textbf{3.} Use the result of Question 2 to prove that there does not exist a rational nu
mber whose square is 5. \\ 

Suppose for contradiction there does exist a rational number $q$ such that $q^2 = 5$.
We write $q$ as a fraction in lowest terms, $q = a/b$ so that $a$ and $b$ are integers 
such that $(a,b) = 1$. Now, $q^2 = 5 \Rightarrow a^2/b^2 = 5 \Rightarrow a^2 = 5b^2 
\Rightarrow a^2$ is divisible by 5. It follows that $a$ much be divisible by 5. \\
Thus, $a = 5a_1^2 = b^2 \Rightarrow b^2$ is divisible by $5 \Rightarrow b$ is 
divisible by 5 as above. \\
Finally, 5 is a common factor of $a$ and $b$ so that $(a,b) \not= 1$. Thus, we reach 
a contradiction to the choice we made for $a$ and $b$. \\
So, there does not exist a rational number whose square is 5. \\

%----------------------------------------------------------------------------------------
%	PROBLEM #6
%----------------------------------------------------------------------------------------
\textbf{6.} Use the Euclidean algorithm to find the greatest common divisors of (i) 165 
and 252, (ii) 4284 and 3480. \\ 

\textbf{(i)} 
\begin{align*}
    (252,165) &\\
    252 &= 165(1) + 87 \\
    165 &= 87(1) + 78 \\
    87 &= 78(1) + 9 \\
    78 &= 9(9) + 6 \\
    9 &= 6(1) + 3 \\
    6 &= 3(2) + 0 \\
    gcd(252,165) &= 3
\end{align*}

\textbf{(ii)}
\begin{align*}
    (4284,3480) & \\
    4284 &= 3480(1) + 804 \\
    3480 &= 804(4) + 264 \\
    804 &= 264(3) + 12 \\
    264 &= 12(22) + 0 \\
    gcd(4284,3480) &= 12
\end{align*}

%----------------------------------------------------------------------------------------
%	PROBLEM #7
%----------------------------------------------------------------------------------------
\textbf{7.} Let $u_n$ be the $n$th Fibonacci number (for the definition see Definition 5.
4.2). Prove that the Euclidean algorithm takes precisely $n$ steps to prove that $gcd(u_{n
+1},u_n) = 1$ \\

Proof by Induction \\
Base Case: (n = 1), $F_1 = 1$ and $F_2 = 1$. Thus, $gcd(F_1,F_2) = gcd(1,1) = 1$ \\

Inductive Step: Need to show that for all $k \ge 2$, if $gcd(u_k,u_{k-1}) = 1$, then 
$gcd(u_{k+1},u_k)$. So suppose that $gcd(u_k,u_{k-1}) = 1$ for some integer $k \ge 2$.
Need to prove that $gcd(u_{k},u_{k+1}) = 1$ \\
\begin{align*}
    gcd(u_k,u_{k+1}) &= gcd(\text{remainder}(u_{k+1},u_k),u_k) \\ 
    &= gcd(\text{remainder}(u_{k-1} + u_k, u_k),u_k) \text{ By def of fibonacci numbers} \\
    &= gcd(u_{k-1},u_k) \\
    &= 1
\end{align*}
In the third step, we can observe that $u_{k-1} < u_k \Rightarrow \text{ remainder of } 
u_{k-1}$ divided by $u_k$ is just $u_{k-1}$. Thus, by our inductive hypothesis, we can 
conclude that $u_k$ and $u_{k-1}$ are relatively prime and thus our claim holds for $n=k$
\\

%----------------------------------------------------------------------------------------
%	PROBLEM #10
%----------------------------------------------------------------------------------------
\textbf{10.} In each case of Question 6 write the greatest common divison as an integral 
linear combination of the two numbers. \\

\textbf{(i)}
\begin{align*}
    3 &= 9(1) - 6(1) \\
    &= 9(1) - 1(78(1) - 9(1)) \\
    &= 9(1) + 78(-1) + 9(8) \\
    &= 78(-1) + 9(9) \\
    &= 78(-1) + 9(87(1)-78(1)) \\
    &= 78(-1) + 87(9) + 78(-9) \\
    &= 87(9) + 78(-10) \\
    &= 87(9) - 10(165(1) - 87(1)) \\
    &= 87(9) + 165(-10) + 87(10) \\
    &= 165(-10) + 87(19) \\
    &= 165(-10) + 19(252(1) - 165(1)) \\
    &= 165(-10) + 252(19) + 165(-19) \\
    &= 252(19) + 165(-29) \\
\end{align*}

\textbf{(ii)}
\begin{align*}
    12 &= 804(1) - 264(3) \\
    &= 804(1) - 3(3480(1) - 804(4)) \\
    &= 804(1) + 3480(-3) + 804(12) \\
    &= 3480(-3) + 804(13) \\
    &= 3480(-3) + 13(4284(1) - 3480(1)) \\
    &= 3480(-3) + 4284(13) + 3480(-13) \\
    &= 4284(13) + 3480(-16)
\end{align*}


\end{document}
