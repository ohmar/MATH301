\documentclass[paper=letter, fontsize=11pt]{scrartcl} % Letter paper and 11pt font size

\usepackage{amstext, amsmath, amssymb}
\usepackage[T1]{fontenc} % Use 8-bit encoding that has 256 glyphs
\usepackage[english]{babel} % English language/hyphenation
\usepackage{amsmath,amsfonts,amsthm} % Math packages

\usepackage{fancyhdr} % Custom headers and footers
\pagestyle{fancyplain} % Makes all pages in the document conform to the custom headers and footers
\fancyhead{} % No page header - if you want one, create it in the same way as the footers below
\fancyfoot[L]{} % Empty left footer
\fancyfoot[C]{} % Empty center footer
\fancyfoot[R]{\thepage} % Page numbering for right footer
\renewcommand{\headrulewidth}{0pt} % Remove header underlines
\renewcommand{\footrulewidth}{0pt} % Remove footer underlines
\setlength{\headheight}{13.6pt} % Customize the height of the header
\setlength\parindent{0pt} % Removes all indentation from paragraphs

%----------------------------------------------------------------------------------------
%	TITLE SECTION
%----------------------------------------------------------------------------------------

\newcommand{\horrule}[1]{\rule{\linewidth}{#1}} % Create horizontal rule command with 1 argument of height

\title{	
\normalfont \normalsize 
\textsc{San Francisco State University} \\ [25pt]
\horrule{0.5pt} \\[0.4cm] % Thin top horizontal rule
\huge MATH 301 Assignment 3  \\ % The assignment title
\horrule{2pt} \\[0.5cm] % Thick bottom horizontal rule
}

\author{Omar Sandoval}

\date{\normalsize\today}

\begin{document}

\maketitle
%----------------------------------------------------------------------------------------
%	PROBLEM 14
%----------------------------------------------------------------------------------------
\textbf{14.} Prove Bernoulli's inequality
\begin{align*}
	(1+x)^n \ge 1 + nx
\end{align*}
for all non-negative integers n and real numbers $x > -1$. \\
--\\
Proof by Induction\\
I) Base Case \\
\begin{align*}
	\text{For } n &= 0, \\
	(1 + x)^0 &\ge 1 + 0 \cdot x \\
	1 &\ge 1 \\
	&\text{Thus, base case holds.} 
\end{align*}

II) Inductive Step \\
Suppose for all non-negative integer \textit{k} and real number $x > -1$, $P(k) \Rightarrow P(k+1).$ \\
$P(k): (1 + x)^k \ge 1 + kx.$ \\
$P(k+1): (1 + x)^{k+1} \ge 1 + (k + 1)x.$ \\

Consider $P(k+1)$: \\
NTS: $(1 + x)^{k+1} \ge 1 + (k + 1)x.$. \\
Notice $x > -1 = 1 + x > 0$ \\
Also, $k \ge 0$ and $x^2 \ge 0$ $\rightarrow$ $xk^2 \ge 0$. \\
By the inductive definition, $(1 + x)^{k+1} = (1 + x)(1 + x)^k.$ \\
By the inductive hypothesis, $(1 + x)^{k+1} \ge 1 + kx \rightarrow (1 + x)(1 + x)^k.$ \\
$ = (1 + x)^{k+1} \ge (1 + kx)(1 + x)$ \\
$ = 1 + x + kx + kx^2$ \\
$ = 1 + (k + 1)x + kx^2 \ge 1 + (k + 1)x + 0$ \\
$ = 1 + (k + 1)x$, since $kx^2 \ge 0$.\\
So, $(1 + x)^{k+1} \ge 1 + (k+1)x$.\\
\\
Finally, through induction on $n$, $(1 + x)^n \ge 1 + nx$ for all non-negative $n$ and real numbers $x > -1$.
\\
\\
%----------------------------------------------------------------------------------------
%	PROBLEM 15
%----------------------------------------------------------------------------------------
\textbf{15.} For which non-negative integer values of \textit{n} is $n! \ge 3^n$? \\
-- \\
The proposition holds for $n = 0$ and integers which are greater than or equal to 7. \\
When $n = 0$, $n! = 0! = 1$, also, $3^n = 3^0 = 1$. \\
Clearly, $1 \ge 1$. \\

Proof by Induction for $n \ge 7$. \\
I) Base Case
$n = 7$ \\
$n! = 7! = 5040$ \\
$3^n = 3^7 = 2187$ \\
$ 5040 \ge 2187$ \\

II) Inductive Step \\
Suppose $P(k) \Rightarrow P(k + 1).$ \\
$P(k): k! \ge 3^k$ for some integer $k \ge 7$. \\
$P(k+1): (k+1)! \ge 3^{k+1}$ \\

$3k! \ge 3 \cdot 3k$ \\
$(k + 1)! = (k + 1)k!$ by our inductive definition. \\
$(7 + 1)k! = 8k! \ge 3k! \ge 3 \cdot 3^k = 3^{k+1}$ \\
Finally, by induction on $n$, $n! \ge 3^n$ for all integers $n \ge 7$.
\\

%----------------------------------------------------------------------------------------
%	PROBLEM 16
%----------------------------------------------------------------------------------------
\textbf{16.} Prove by induction on \textit{n} that
\begin{align*}
	\sum_{i=1}^{n} \frac{1}{i(i+1)} = \frac{n}{n+1}
\end{align*}
for all positive integers \textit{n}. \\
--\\

Proof by Induction \\

I)Base Case \\

For $n = 1$, \\
$\sum_{i=1}^1 \frac{1}{i(i+1)} \ \frac{1}{1(1+1} = \frac{1}{2}$ 
and, $\frac{n}{n+1} = \frac{1}{1+1} \frac{1}{2}$ \\

Clearly, base case holds.\\

II) Inductive Step \\
Suppose $P(k) \Rightarrow P(k + 1)$. \\

$P(k): \sum_{i=1}^{k} \frac{1}{i(i+1)} = \frac{k}{k+1}$ \\
$P(k+1): \sum_{i=1}^{k+1} \frac{1}{i(i+1)} = \frac{k+1}{k+2}$ \\
By the inductive definition and hypothesis, \\
$\sum_{i=1}^{k+1} \dfrac{1}{i(i+1)} = \sum_{i=1}^{k}
 \dfrac{1}{i(i+1)} + \dfrac{1}{(k+1)(k+2)}$ \\
$= \dfrac{k}{k+1}+\dfrac{1}{(k+1)(k+2)}$ \\
$= \dfrac{k^2 + 2k + 1}{(k+1)(k+2)}$ \\
$= \dfrac{(k+1)^2}{(k+1)(k+2)}$ \\
$= \dfrac{k+1}{k+2}$ \\

Finally, by induction on $n$, $\sum_{i=1}^n \dfrac{1}{i(i+1)} = \dfrac{n}{n+1}$

%----------------------------------------------------------------------------------------
%	PROBLEM 17
%----------------------------------------------------------------------------------------
\textbf{17.} For a positive integer \textit{n} the number $a_n$ is defined inductively by
\begin{align*}
	a_1 &= 1, \\
	a_{k+1} &= \frac{6a_k + 5}{a_k + 2}, \text{  for } k \text{ a positive integer.} \\
\end{align*}
-- \\
(i) $a_n > 0$ \\
I) Base Case \\

For $n = 1$, \\
$a_n = a_1 = 1 > 0$ \\
Base case holds. \\

II) Inductive Step\\

Suppose $P(k) \Rightarrow P(k+1)$ \\
$P(k): a_k > 0$ \\
$P(k+1): a_{k+1} = \dfrac{6a_k + 5}{a_k + 2} > 0$.\\

Using our inductive hypothesis, \\
$a_k > 0 \rightarrow a_k + 2 > 0$ and $a_k > 0 \rightarrow 
2a_k > a_k \rightarrow 6a_k > a_k \rightarrow 6a_k + 5 > a_k + 2 > 0.$. \\
So, $\dfrac{6a_k + 5}{a_k + 2} > 1 > 0 \rightarrow \dfrac{6a_k + 5}{a_k + 2} > 0$. \\

Finally, by induction on $n$, $a_n < 5$ for all positive integers $n$.

(ii) $a_n < 5$. \\
I) Base Case

For $n = 1$, \\
$a_n = a_1 = 1 < 5$. \\
Base case holds.

II) Inductive Step \\

Suppose $P(k) \Rightarrow P(k+1)$.\\
$P(k): a_k < 5$ \\
$P(k+1): a_{k+1} = \dfrac{6a_k + 5}{a_k + 2} < 5$. \\

Using our inductive hypothesis, \\
$a_k < 5 \rightarrow 6a_k < 5a_k + 5 \rightarrow 6a_k + 5 < 5a_k + 10$ \\ 
$= 5(a_k + 2) \rightarrow 6a_k + 5 < 5(a_k + 2)$.
So, $\dfrac{6a_k + 5}{a_k + 2} < 5$.

Finally, by induction on $n$, $a_n < 5$ for all positive integers $n$.
\\

%----------------------------------------------------------------------------------------
%	PROBLEM 19
%----------------------------------------------------------------------------------------
\textbf{19.} Prove that 
\begin{align*}
	\prod_{i=1}^n (1 - \frac{1}{i^2}) = \frac{n + 1}{2n}
\end{align*} \\
for integers $n \ge 2$.\\
-- \\

I) Base Case \\

For $n = 2$ \\
$\prod\limits_{i=1}^2 (1 - \dfrac{1}{i^2}) = 1 - \dfrac{1}{2^2})$
$= \dfrac{3}{4}$ \\ 

$ \dfrac{n+1}{2n} = \dfrac{2+1}{2(2)} = \dfrac{3}{4} $.\\
Base case holds. \\

II) Inductive Step \\

Suppose that $P(k) \Rightarrow P(k+1)$\\
$P(k): \prod\limits_{i=1}^k (1 - \dfrac{1}{i^2}) = \dfrac{k + 1}{2k}$ \\
$P(k+1): \prod\limits_{i=1}^{k+1} (1 - \dfrac{1}{i^2})
 = \dfrac{(k+1) + 1}{2(k+1)} = \dfrac{k + 2}{2k + 2}$ \\

\begin{align*}
	\prod_{i=1}^{k+1} (1 - \frac{1}{i^2}) &=
	[\prod_{i=1}^k (1- \dfrac{1}{i^2})] \cdot (1 - \dfrac{1}{(k+1)^2}) \\
	&= \dfrac{k+1}{2k} \cdot (1 - \dfrac{1}{(k+1)^2}) \\
	&= \dfrac{(k+1)[(k+1)^2 - 1]}{2k(k+1)^2} \\
	&= \dfrac{(k+1)^2 - 1}{2k(k+1)} \\
	&= \dfrac{k^2 + 2k}{2k(k+1)} \\
	&= \dfrac{k(k+2)}{2k(k+1)} \\
	&= \dfrac{k+2}{2k+2}
\end{align*}

Finally, by induction on n, $\prod\limits_{i=1}^n (1 - \frac{1}{i^2})
= \frac{n + 1}{2n}$, for all integers $n \ge 2$.

\end{document}
