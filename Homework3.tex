\documentclass[paper=letter, fontsize=11pt]{scrartcl} % Letter paper and 11pt font size

\usepackage{amstext, amsmath, amssymb}
\usepackage[T1]{fontenc} % Use 8-bit encoding that has 256 glyphs
\usepackage[english]{babel} % English language/hyphenation
\usepackage{amsmath,amsfonts,amsthm} % Math packages

\usepackage{fancyhdr} % Custom headers and footers
\pagestyle{fancyplain} % Makes all pages in the document conform to the custom headers and footers
\fancyhead{} % No page header - if you want one, create it in the same way as the footers below
\fancyfoot[L]{} % Empty left footer
\fancyfoot[C]{} % Empty center footer
\fancyfoot[R]{\thepage} % Page numbering for right footer
\renewcommand{\headrulewidth}{0pt} % Remove header underlines
\renewcommand{\footrulewidth}{0pt} % Remove footer underlines
\setlength{\headheight}{13.6pt} % Customize the height of the header
\setlength\parindent{0pt} % Removes all indentation from paragraphs

%----------------------------------------------------------------------------------------
%	TITLE SECTION
%----------------------------------------------------------------------------------------

\newcommand{\horrule}[1]{\rule{\linewidth}{#1}} % Create horizontal rule command with 1 argument of height

\title{	
\normalfont \normalsize 
\textsc{San Francisco State University} \\ [25pt]
\horrule{0.5pt} \\[0.4cm] % Thin top horizontal rule
\huge MATH 301 Assignment 3  \\ % The assignment title
\horrule{2pt} \\[0.5cm] % Thick bottom horizontal rule
}

\author{Omar Sandoval}

\date{\normalsize\today}

\begin{document}

\maketitle
%----------------------------------------------------------------------------------------
%	PROBLEM 14
%----------------------------------------------------------------------------------------
\textbf{14.} Prove Bernoulli's inequality
\begin{align*}
	(1+x)^n \ge 1 + nx
\end{align*}
for all non-negative integers n and real numbers $x > -1$. \\

%----------------------------------------------------------------------------------------
%	PROBLEM 15
%----------------------------------------------------------------------------------------
\textbf{15.} For which non-negative integer values of \textit{n} is $n! \ge 3^n$? \\

%----------------------------------------------------------------------------------------
%	PROBLEM 16
%----------------------------------------------------------------------------------------
\textbf{16.} Prove by induction on \textit{n} that
\begin{align*}
	\sum_{i=1}^{n} \frac{1}{i(i+1)} = \frac{n}{n+1}
\end{align*}
for all positive integers \textit{n}. \\

%----------------------------------------------------------------------------------------
%	PROBLEM 17
%----------------------------------------------------------------------------------------
\textbf{17.} For a positive integer \textit{n} the number $a_n$ is defined inductively by
\begin{align*}
	a_1 &= 1, \\
	a_{k+1} &= \frac{6a_k + 5}{a_k + 2}, \text{  for } k \text{ a positive integer.} \\
\end{align*}

%----------------------------------------------------------------------------------------
%	PROBLEM 19
%----------------------------------------------------------------------------------------
\textbf{19.} Prove that 
\begin{align*}
	\prod_{i=1}^n (1 - \frac{1}{i^2}) = \frac{n + 1}{2n}
\end{align*}
for integers $n \ge 2$.

\end{document}
