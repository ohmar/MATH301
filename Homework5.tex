\documentclass[paper=letter, fontsize=11pt]{scrartcl} % Letter paper and 11pt font size

\usepackage{amstext, amsmath, amssymb, cases, graphicx}
\usepackage[T1]{fontenc} % Use 8-bit encoding that has 256 glyphs
\usepackage[english]{babel} % English language/hyphenation
\usepackage{amsmath,amsfonts,amsthm} % Math packages

\usepackage{fancyhdr} % Custom headers and footers
\pagestyle{fancyplain} % Makes all pages in the document conform to the custom headers and footers
\fancyhead{} % No page header
\fancyfoot[L]{} % Empty left footer
\fancyfoot[C]{} % Empty center footer
\fancyfoot[R]{\thepage} % Page numbering for right footer
\renewcommand{\headrulewidth}{0pt} % Remove header underlines
\renewcommand{\footrulewidth}{0pt} % Remove footer underlines
\setlength{\headheight}{13.6pt} % Customize the height of the header
\setlength\parindent{0pt} % Remove all indentation from paragraps.

%----------------------------------------------------------------------------------------
%	TITLE SECTION
%----------------------------------------------------------------------------------------

\newcommand{\horrule}[1]{\rule{\linewidth}{#1}} % Create horizontal rule command with 1 argument of height

\title{	
\normalfont \normalsize 
\textsc{San Francisco State University} \\ [25pt]
\horrule{0.5pt} \\[0.4cm] % Thin top horizontal rule
\huge MATH 301 Assignment 5  \\ % The assignment title
\horrule{2pt} \\[0.5cm] % Thick bottom horizontal rule
}

\author{Omar Sandoval}

\date{\normalsize\today}

\begin{document}

\maketitle

%----------------------------------------------------------------------------------------
%	PROBLEM #14
%----------------------------------------------------------------------------------------
\textbf{14.} Define functions $f$ and $g$: $\mathbb{R} \rightarrow \mathbb{R}$ by $f(x) =
x^2$ and $g(x) = x^2-1$. Find the functions $f \circ f$, $f \circ g$, $g \circ f$, $g
\circ g$. \\
List the elements of the set ${x \in \mathbb{R} | f(g(x)) = g(f(x))}$ \\

$(f \circ f) = x^4$ \\
$(f \circ g) = (x^2 -1)^2$ \\
$(g \circ f) = x^2 -1$ \\
$(g \circ g) = (x^2 - 1)^2 -1$ \\

\begin{align*}
    f(g(x)) = g(f(x)) &\Leftrightarrow (x^2 - 1)^2 - 1 \\
    &\Leftrightarrow x^4 - 2x^2 + 1 = x^4 - 1 \\
    &\Leftrightarrow -2x^2 + 2 = 0 \\
    &\Leftrightarrow x^2 = 1 \\
    &\Leftrightarrow x = \pm 1
\end{align*}
So, ${x \in \mathbb{R} | f(g(x)) = g(f(x))}$. \\

%----------------------------------------------------------------------------------------
%	PROBLEM #16 (ii, iv, v, vi)
%----------------------------------------------------------------------------------------
\textbf{16.} Determine which of the following functions $f_i: \mathbb{R} \rightarrow
\mathbb{R}$: are injective, which are surjective and which are bijective. Write down an
inverse function of each of the bijections. \\

\textbf{(ii)} $f_2(x) = x^3$; \\

$f_2$ is bijective. \\
So, since $y=f_2(x) \Leftrightarrow y = x^3 \Leftrightarrow x = y^{1/3}$ for $x, y \in 
\mathbb{R}$, we can see that each element of $\mathbb{R}$ has one pre-image under $f_2$.
Clearly, $f_2$ is a bijection. \\

$f_2^{-1}(x) = x^{1/3}$ \\

\textbf{(iv)} $f_4(x) = x^3 - 3x^2 + 3x - 1$; \\

$f_4$ is bijective. \\ 
So, since $f_4(x) \Leftrightarrow y = x^2 - 3x^2 + 3x - 1 \Leftrightarrow x = y^{1/3} + 1$
for $x, y \in \mathbb{R}$, we cam see that each element of $\mathbb{R}$ has one pre-image
under $f_4$. Clearly, $f_4$ is a bijection. \\
$f_4^{-1}(x) = x^{1/3} + 1$ \\

\textbf{(v)} $f_5(x) = e^x$; \\

$f_5$ is injective and not surjective. \\
$f_5^{-1}=log(x)$ \\
$e^x$ never takes on any negative value, thus, it cannot be surjective.

\textbf{(vi)}

\[f_6(x) = \left\{
  \begin{array}{lr}
    x^2 & \text{ if } x \ge 0\\
    -x^2 & \text{ if } x \le 0
  \end{array}
\right.
\]

$f_6$ is bijective.
\[f_6^{-1}(x) = \left\{
  \begin{array}{lr}
    x^{1/2} & \text{ if } x \ge 0\\
    -\sqrt{-x} & \text{ if } x \le 0
  \end{array}
\right.
\]


%----------------------------------------------------------------------------------------
%	PROBLEM #17
%----------------------------------------------------------------------------------------
\textbf{17.} Functions $f:\mathbb{R} \rightarrow \mathbb{R}$ and $g: \mathbb{R} \rightarrow
\mathbb{R}$ are defined as follows. \\

\begin{numcases}{f(x)=}
    x + 2 & if $x < -1$,
    \\
    -x & if $-1 \le x \le 1$,
    \\
    x - 2 & if $x > 1$.
\end{numcases}

\begin{numcases}{g(x)=}
    x - 2 & if $x < -1$,
    \\
    -x & if $-1 \le x \le 1$,
    \\
    x + 2 & if $x > 1$.
\end{numcases}
\\
Find the functions $f \circ g$ and $g \circ f$. Is $g$ the inverse of the function $f$?
Is $f$ injective or surjective? How about $g$? Sketch and compare the graphs of these functions.
\\

$f \circ g = I_\mathbb{R}$ \\
\[g \circ f(x) = \left\{
    \begin{array}{lr}
	x & \text{ if } x < -3, \\
	-x - 2 & \text{ if } -3 \le x < -1, \\
	x if -1 & \text{ if } \le x \le 1, \\
	2 - x & \text{ if } 1 < x \le 3, \\
	x & \text{ if } x > 3.
    \end{array}
\right.
\]

$g$ is not the inverse function of $f$. We can see that $f(-2) = f(0) = f(2) = 0$.
If $g$ were the inverse function of $f$, we would have $g(0) = -2, g(0) = 0, g(0) = 2.$
Which is not possible since $g$ is single-valued. $f$ is not injective and therefore
not invertible. \\
$f$ is surjective but not injective, and $g$ is injective but not surjective. \\

The graphs of $f(x)$ and $g(x)$ follow:

\includegraphics[scale=0.5]{graphs.png}
\\

%----------------------------------------------------------------------------------------
%	PROBLEM #18
%----------------------------------------------------------------------------------------
\textbf{18.} Suppose that $f: X \rightarrow Y$ and $g: Y \rightarrow Z$ are surjections. Prove that the
composite $g \circ f: X \rightarrow Z$ is a surjection. \\

Let $z \in Z$. Since $g$ is a surjection, there exists a $y \in Y$ such that $g(y)=\mathbb{z}$.
But $f$ is a surjection, then there exists an $x \in X$ such that $f(x) = y$. Therefore, 
$(g \circ f)(x) = \mathbb{z}$, and $(g \circ f): X \rightarrow Z$ is then a surjection. \\

\end{document}
