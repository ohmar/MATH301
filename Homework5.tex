\documentclass[paper=letter, fontsize=11pt]{scrartcl} % Letter paper and 11pt font size

\usepackage{amstext, amsmath, amssymb, cases}
\usepackage[T1]{fontenc} % Use 8-bit encoding that has 256 glyphs
\usepackage[english]{babel} % English language/hyphenation
\usepackage{amsmath,amsfonts,amsthm} % Math packages

\usepackage{fancyhdr} % Custom headers and footers
\pagestyle{fancyplain} % Makes all pages in the document conform to the custom headers and footers
\fancyhead{} % No page header
\fancyfoot[L]{} % Empty left footer
\fancyfoot[C]{} % Empty center footer
\fancyfoot[R]{\thepage} % Page numbering for right footer
\renewcommand{\headrulewidth}{0pt} % Remove header underlines
\renewcommand{\footrulewidth}{0pt} % Remove footer underlines
\setlength{\headheight}{13.6pt} % Customize the height of the header
\setlength\parindent{0pt} % Remove all indentation from paragraps.

%----------------------------------------------------------------------------------------
%	TITLE SECTION
%----------------------------------------------------------------------------------------

\newcommand{\horrule}[1]{\rule{\linewidth}{#1}} % Create horizontal rule command with 1 argument of height

\title{	
\normalfont \normalsize 
\textsc{San Francisco State University} \\ [25pt]
\horrule{0.5pt} \\[0.4cm] % Thin top horizontal rule
\huge MATH 301 Assignment 5  \\ % The assignment title
\horrule{2pt} \\[0.5cm] % Thick bottom horizontal rule
}

\author{Omar Sandoval}

\date{\normalsize\today}

\begin{document}

\maketitle

%----------------------------------------------------------------------------------------
%	PROBLEM #14
%----------------------------------------------------------------------------------------
\textbf{14.} Define functions $f$ and $g$: $\mathbb{R} \rightarrow \mathbb{R}$ by $f(x) =
x^2$ and $g(x) = x^2-1$. Find the functions $f \circ f$, $f \circ g$, $g \circ f$, $g
\circ g$. \\
List the elements of the set ${x \in \mathbb{R} | f(g(x)) = g(f(x))}$ \\

%----------------------------------------------------------------------------------------
%	PROBLEM #16 (ii, iv, v, vi)
%----------------------------------------------------------------------------------------
\textbf{16.} Determine which of the following functions $f_i: \mathbb{R} \rightarrow
\mathbb{R}$: are injective, which are surjective and which are bijective. Write down an
inverse function of each of the bijections. \\

\textbf{(ii)} $f_1(x) = x - 1$; \\
\textbf{(iv)} $f_4(x) = x^3 - x$; \\
\textbf{(v)} $f_5(x) = e^x$; \\
\textbf{(vi)}
\[f_6(x) = \left\{
  \begin{array}{lr}
    x^2 & \text{ if } x \ge 0\\
    -x^2 & \text{ if } x \le 0
  \end{array}
\right.
\]

%----------------------------------------------------------------------------------------
%	PROBLEM #17
%----------------------------------------------------------------------------------------
\textbf{17.} Functions $f:\mathbb{R} \rightarrow \mathbb{R}$ and $g: \mathbb{R} \rightarrow
\mathbb{R}$ are defined as follows. \\

\begin{numcases}{f(x)=}
    x + 2 & if $x < -1$,
    \\
    -x & if $-1 \le x \le 1$,
    \\
    x - 2 & if $x > 1$.
\end{numcases}

\begin{numcases}{g(x)=}
    x - 2 & if $x < -1$,
    \\
    -x & if $-1 \le x \le 1$,
    \\
    x + 2 & if $x > 1$.
\end{numcases}
\\
Find the functions $f \circ g$ and $g \circ f$. Is $g$ the inverse of the function $f$?
Is $f$ injective or surjective? How about $g$? Sketch and compare the graphs of these functions.
\\

%----------------------------------------------------------------------------------------
%	PROBLEM #18
%----------------------------------------------------------------------------------------
\textbf{18.} Suppose that $f: X \rightarrow Y$ and $g: Y \rightarrow Z$ are surjections. Prove that the
composite $g \circ f: X \rightarrow Z$ is a surjection.

\end{document}
