\documentclass[paper=letter, fontsize=11pt]{scrartcl} % Letter paper and 11pt font size

\usepackage{amstext, amsmath, amssymb}
\usepackage[T1]{fontenc} % Use 8-bit encoding that has 256 glyphs
\usepackage[english]{babel} % English language/hyphenation
\usepackage{amsmath,amsfonts,amsthm} % Math packages

\usepackage{fancyhdr} % Custom headers and footers
\pagestyle{fancyplain} % Makes all pages in the document conform to the custom headers and footers
\fancyhead{} % No page header - if you want one, create it in the same way as the footers below
\fancyfoot[L]{} % Empty left footer
\fancyfoot[C]{} % Empty center footer
\fancyfoot[R]{\thepage} % Page numbering for right footer
\renewcommand{\headrulewidth}{0pt} % Remove header underlines
\renewcommand{\footrulewidth}{0pt} % Remove footer underlines
\setlength{\headheight}{13.6pt} % Customize the height of the header
\setlength\parindent{0pt} % Removes all indentation from paragraphs

%----------------------------------------------------------------------------------------
%	TITLE SECTION
%----------------------------------------------------------------------------------------

\newcommand{\horrule}[1]{\rule{\linewidth}{#1}} % Create horizontal rule command with 1 argument of height

\title{	
\normalfont \normalsize 
\textsc{San Francisco State University} \\ [25pt]
\horrule{0.5pt} \\[0.4cm] % Thin top horizontal rule
\huge MATH 301 Assignment 2  \\ % The assignment title
\horrule{2pt} \\[0.5cm] % Thick bottom horizontal rule
}

\author{Omar Sandoval}

\date{\normalsize\today}

\begin{document}

\maketitle

%----------------------------------------------------------------------------------------
%	PROBLEM 8
%----------------------------------------------------------------------------------------
\textbf{8.} Prove the following statements concerning a real number \textit{x}. \\
(i) $x^2 - x - 2 = o \Leftrightarrow x = -1$ or $x = 2$.\\
(ii) $x^2 - x - 2 > 0 \Leftrightarrow x < -1$ or $x > 2$.
\\
-- \\
Proving $\rightarrow$: \\
\begin{align*}
	x^2 - x - 2 = 0 &\rightarrow (x-2)(x+1) = 0. \\
	&\rightarrow (x-2) = 0 \text{ or } (x+1) = 0. \\
	&\rightarrow x = 2 \text{ or } x = -1. \\
\end{align*}

So, $x^2 - x - 2 = 0 \rightarrow x = 2 or x = -1.$ \\\\
Proving $\leftarrow$: \\

If $x = 2$, \\
	
	$x^2 - x - 2 = 2^2 - 2 - 2 = 0.$ \\
	
If $x = -1$, \\
	
	$x^2 - x - 2 = (-1)^2 - (-1) - 2 = 0.$ \\\\
Therefore, $x^2 - x - 2 = 0 \leftrightarrow x = 2$ or $x = -1$.\\
\\

%----------------------------------------------------------------------------------------
%	PROBLEM 9
%----------------------------------------------------------------------------------------
\textbf{9.}	Prove by contradiction that there does not exist a largest integer. [Hint: Observe that for any integer \textit{n} there is a greater one, say \textit{n} + 1. So begin your proof:  
Suppose for contradiction that there is a largest integer. Let this integer be \textit{n}.$ \cdot$]
\\
-- \\ 
Suppose for contradiction that there is a largest integer. Let this integer be \textit{n}. Notice, $0 < 1 \rightarrow n + 1$. Therefore, $n$, is not the largest integer, since for all $n$, $n + 1 > n$.\\
\\

%----------------------------------------------------------------------------------------
%	PROBLEM 10
%----------------------------------------------------------------------------------------
\textbf{10.} What is wrong with the following proof that 1 is the largest integer? \\

Let \textit{n} be the largest integer. Then, since 1 is an integer we must have $1 \le n$. On the other hand, since $n^2$ is also an integer we must have $n^2 \le n$ from which it follow that $n \le 1$. Thus, since $1 \le n$ and $n \le 1$ we must have $n = 1$. Thus 1 is that largest integer as claimed.
\\

What does this argument prove?
\\
--\\
The initial statement in this proof is false, which we showed in Problem 9. The conclusion is also false since 1 is not the largest integer. This argument tries to prove that if a largest integer existed, that integer would be 1.\\
\\

%----------------------------------------------------------------------------------------
%	PROBLEM 11
%----------------------------------------------------------------------------------------
\textbf{11.} Prove by contradiction that there does not exist a smallest positive real number.
\\
--\\
Suppose for contradiction that there exists a smallest positive real number, $n$. Notice that $ 0 < \frac{1}{2} \rightarrow 0 < \frac{1}{2}n < n.$ So, the integer $\frac{1}{2}n,$ is real and less than $n$. That is a contradiction. Therefore, the original statement holds.\\
\\

%----------------------------------------------------------------------------------------
%	PROBLEM 12
%----------------------------------------------------------------------------------------
\textbf{12.} Prove by induction on \textit{n} that, for all positive integers n, 3 divides $4^n + 5$.
\\
--\\
I) Base Step:\\
For $n = 1, 4^n + 5 = 4^1 + 5 = 9$ which is indeed divisible by 3. \\

II) Inductive Step: Suppose for some positive $k$, $P(k) \Rightarrow P(k+1)$.\\
$P(k): 4^k + 5 = 3q.$\\
$P(k+1): 4^{k+1} + 5 = 3q.$\\

Considering $P(k+1):$
\begin{align*}
	4^{k+1} + 5 &= 4 \cdot 4^k + 5\\
	&= 4(3q - 5) + 5, \text{ by our inductive hypothesis.}\\
	&= 12q - 15 \\
	&= 3(4q - 5) \\
	&= 3p, \text{ where \textit{p} is an integer.}\\
	& \text{Thus, 3 divides } 4^{k+1} + 5.
\end{align*}
Finally, by induction on $n$, 3 divides $4^n + 5$ for all positive integers $n$.\\
\\

%----------------------------------------------------------------------------------------
%	PROBLEM 13
%----------------------------------------------------------------------------------------
\textbf{13.} Prove by induction on \textit{n} that $n! > 2^n$ for all integers $n$ such that $n \ge 4$.
\\
--\\
I) Base Step: $P(4) = 4! > 2^4 = 24 > 16.$\\
Clearly, base case holds. \\

II) Inductive step: Suppose for some positive $k \ge 4$, $P(k) \Rightarrow P(k+1)$ \\
$P(k): k! > 2^k$ \\
$P(k+1): (k+1)! > 2^{k+1}$ \\
Considering $P(k+1):$
\begin{align*}
	2k! &> 2 \cdot 2^k \\
	&> 2^{k+1}, \text{ by our inductice hypothesis.}\\
	\\
	(k+1)! &= (k+1)k! \\
	(k+1)k! &\ge (4+1)k!, \text{ because k >= 4}\\ 
	&= 5k! > 2k! > 2 \cdot 2^k = 2^{k+1} \\
\end{align*}
Finally, by induction on $n$, $n! > 2^n$ for all integers greater than or equal to 4.

\end{document}