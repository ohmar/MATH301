\documentclass[paper=letter, fontsize=11pt]{scrartcl} % Letter paper and 11pt font size

\usepackage{amstext, amsmath, amssymb}
\usepackage[T1]{fontenc} % Use 8-bit encoding that has 256 glyphs
\usepackage[english]{babel} % English language/hyphenation
\usepackage{amsmath,amsfonts,amsthm} % Math packages

\usepackage{fancyhdr} % Custom headers and footers
\pagestyle{fancyplain} % Makes all pages in the document conform to the custom headers and footers
\fancyhead{} % No page header
\fancyfoot[L]{} % Empty left footer
\fancyfoot[C]{} % Empty center footer
\fancyfoot[R]{\thepage} % Page numbering for right footer
\renewcommand{\headrulewidth}{0pt} % Remove header underlines
\renewcommand{\footrulewidth}{0pt} % Remove footer underlines
\setlength{\headheight}{13.6pt} % Customize the height of the header
\setlength\parindent{0pt} % Removes all indentation from paragraphs

%----------------------------------------------------------------------------------------
%	TITLE SECTION
%----------------------------------------------------------------------------------------

\newcommand{\horrule}[1]{\rule{\linewidth}{#1}} % Create horizontal rule command with 1 argument of height

\title{	
\normalfont \normalsize 
\textsc{San Francisco State University} \\ [25pt]
\horrule{0.5pt} \\[0.4cm] % Thin top horizontal rule
\huge MATH 301 Homework 10  \\ % The assignment title
\horrule{2pt} \\[0.5cm] % Thick bottom horizontal rule
}

\author{Omar Sandoval}

\date{\normalsize\today}

\begin{document}

\maketitle

%----------------------------------------------------------------------------------------
%	PROBLEM 5
%----------------------------------------------------------------------------------------
\textbf{5.} Find a test similar to those in the previous questions for divisibility of
99. \\

Suppose a positive integer is written in decimal notation as $n = a_k a_{k-1} \dots a_1
a_0$ where $0 \le a_i \le 9$. Prove that $n$ is divisible by 99 if and only if the sum 
of it's digits $a_k + a_{k-1} + \dots + a_1 + a_0$ is also divisible by 99. \\

It suffices to check that both $n$ and the sum of it's digits are congruent modulo 9.
This follows from another application of the binomial formula, for $(99+1)^n$. We can
then write $n = \sum_{i=0}^k a_i 100^i$. So;
\begin{center}
    $\sum\limits_{i=0}^k a_i 100^i = \sum\limits_{i=0}^k a_i (99+1)^i = 
    \sum\limits_{i=0}^k a_i \sum\limits_{j=0}^i \dbinom{i}{j} 99^j 
    \equiv 99 \sum\limits_{i=0}^k a_i$
\end{center}
Since $100 \equiv 1 \text{ mod } 99$, it follows that $100^k \equiv 1 \text{ mod } 99$.
Thus, $n \equiv 100^ka_{k} + 100^{k-1}a_{k-1} + \dots + 100^1a_1 + a_0 \equiv a_k + 
\dots + a_0 \text{ mod } 99.$ \\
Thus, the congruency follows since there is only one term in the binomial equation that
is not a multiple of 99, that is, $j = 0$.
\\

%----------------------------------------------------------------------------------------
%	PROBLEM 8
%----------------------------------------------------------------------------------------
\textbf{8.} Prove that the Fibonacci number $u_n$, (see Definition 5.4.2) is divisible 
by 3 if and only if $n$ is divisible by 4. \\

We can see first which Fibonacci numbers are congruent to 0 mod 3, 1 mod 3, 2 mod
3, and which are divisible by 3. \\

\[
\begin{cases}
    f_n \equiv 0 (\text{mod } 3) & \text{ if } n \equiv 0 \text{ or } 4 (\text{mod } 8); \\
    f_n \equiv 1 (\text{mod } 3) & \text{ if } n \equiv 1, 2, \text{ or } 4 (\text{mod } 8); \\
    f_n \equiv 2 (\text{mod } 3) & \text{ if } n \equiv 3, 5, \text{ or } 6 (\text{mod } 8).
\end{cases}
\]

Being divisible by 4 is the same as begin congruent to 0 or 4 mod 8. \\
\textit{Base Cases}: n = 1, 2. \\
$n = 1, 2$: $f_1 = f_2 = 1$. \\

\textit{Inductive Step}: Suppose the proposition is true for $m > 2$, and all values
of $n$ up to $m - 1$. We need to show it is also true for $n = m$. We have then have
the following 8 cases. \\

If $m \equiv 0$ (mod 8), then $m - 2 \equiv 6$ (mod 8) and $m - 1 \equiv 7$ (mod 8).
Thus, $f_{m-2} \equiv 2$ (mod 3) and $f_{m-1} \equiv 1$ (mod 3), so
$f_m = f_{m-2} + f_{m-1} \equiv 2 + 1 \equiv 0$ (mod 3). \\

If $m \equiv 1$ (mod 8), then $m - 2 \equiv 7$ (mod 8) and $m - 1 \equiv 0$ (mod 8).
Thus, $f_{m-2} \equiv 1$ (mod 3) and $f_{m-1} \equiv 0$ (mod 3), so
$f_m = f_{m-2} + f_{m-1} \equiv 1 + 0 \equiv 1$ (mod 3). \\

If $m \equiv 2$ (mod 8), So, $f_{m-2} \equiv 0$ (mod 3) and $f_{m-1} \equiv 1$ (mod 3), so
$f_m = \equiv 0$ (mod 3). \\

If $m \equiv 3$ (mod 8), So, $f_{m-2} \equiv 1$ (mod 3) and $f_{m-1} \equiv 1$ (mod 3), so
$f_m = f_{m-2} + f_{m-1} \equiv 1 + 1 \equiv 2$ (mod 3). \\

If $m \equiv 4$ (mod 8), So, $f_{m-2} \equiv 1$ (mod 3) and $f_{m-1} \equiv 2$ (mod 3), so
$f_m = f_{m-2} + f_{m-1} \equiv 2 + 1 \equiv 0$ (mod 3). \\

If $m \equiv 5$ (mod 8), So, $f_{m-2} \equiv 2$ (mod 3) and $f_{m-1} \equiv 0$ (mod 3), so
$f_m = f_{m-2} + f_{m-1} \equiv 2 + 0 \equiv 2$ (mod 3). \\

If $m \equiv 6$ (mod 8), So, $f_{m-2} \equiv 0$ (mod 3) and $f_{m-1} \equiv 2$ (mod 3), so
$f_m = f_{m-2} + f_{m-1} \equiv 0 + 2 \equiv 2$ (mod 3). \\

If $m \equiv 7$ (mod 8), So, $f_{m-2} \equiv 2$ (mod 3) and $f_{m-1} \equiv 2$ (mod 3), so
$f_m = f_{m-2} + f_{m-1} \equiv 2 + 2 \equiv 1$ (mod 3). \\


%----------------------------------------------------------------------------------------
%	PROBLEM 12
%----------------------------------------------------------------------------------------
\textbf{12.} Write down the multiplication table for $\mathbb{R}_{15}$. From the table
identify the invertible elements modulo 15 and their inverses. \\

\begin{tabular}{|l|l|l|l|l|l|l|l|l|l|l|l|l|l|l|l|}
\hline
$\mathbb{R}_{15}$ & 0 & 1  & 2  & 3  & 4  & 5  & 6  & 7  & 8  & 9  & 10 & 11 & 12 & 13 & 14 \\ \hline
0               & 0 & 0  & 0  & 0  & 0  & 0  & 0  & 0  & 0  & 0  & 0  & 0  & 0  & 0  & 0  \\ \hline
1               & 0 & 1  & 2  & 3  & 4  & 5  & 6  & 7  & 8  & 9  & 10 & 11 & 12 & 13 & 14 \\ \hline
2               & 0 & 2  & 4  & 6  & 8  & 10 & 12 & 14 & 1  & 3  & 5  & 7  & 9  & 11 & 13 \\ \hline
3               & 0 & 3  & 6  & 9  & 12 & 0  & 3  & 6  & 9  & 12 & 0  & 3  & 6  & 9  & 12 \\ \hline
4               & 0 & 4  & 8  & 12 & 1  & 5  & 9  & 13 & 2  & 6  & 10 & 14 & 3  & 7  & 11 \\ \hline
5               & 0 & 5  & 10 & 0  & 5  & 10 & 0  & 5  & 10 & 0  & 5  & 10 & 0  & 5  & 10 \\ \hline
6               & 0 & 6  & 12 & 3  & 9  & 0  & 6  & 12 & 13 & 9  & 0  & 6  & 12 & 13 & 9  \\ \hline
7               & 0 & 7  & 14 & 6  & 13 & 5  & 12 & 4  & 11 & 3  & 5  & 2  & 9  & 1  & 8  \\ \hline
8               & 0 & 8  & 1  & 9  & 2  & 5  & 3  & 11 & 4  & 12 & 5  & 13 & 6  & 14 & 7  \\ \hline
9               & 0 & 9  & 3  & 12 & 6  & 0  & 9  & 3  & 12 & 6  & 0  & 9  & 3  & 12 & 6  \\ \hline
10              & 0 & 10 & 5  & 0  & 10 & 5  & 0  & 10 & 5  & 0  & 10 & 5  & 0  & 10 & 5  \\ \hline
11              & 0 & 11 & 7  & 3  & 14 & 10 & 6  & 2  & 13 & 9  & 5  & 1  & 12 & 8  & 4  \\ \hline
12              & 0 & 12 & 9  & 6  & 3  & 0  & 12 & 9  & 6  & 3  & 0  & 12 & 9  & 6  & 3  \\ \hline
13              & 0 & 13 & 11 & 9  & 7  & 5  & 3  & 1  & 14 & 12 & 10 & 8  & 6  & 4  & 2  \\ \hline
14              & 0 & 14 & 13 & 12 & 11 & 10 & 9  & 8  & 7  & 6  & 5  & 4  & 3  & 2  & 1  \\ \hline
\end{tabular}
\\

We can see that the invertible elements modulo 15 are 1, 2, 4, 7, 8, 11, 13, 14.
The inverses are 1, 8, 4, 13, 2, 11, 7, 14, in order.\\

%----------------------------------------------------------------------------------------
%	PROBLEM 15
%----------------------------------------------------------------------------------------
\textbf{15.} Prove that 245 is invertible modulo 666 and find its inverse. Hence, solve 
the linear congruences:
\textbf{(i)} $245x \equiv 3 \text{ mod } 666$; \\

We can use the euclidean algorithm to find gcd(245,666). \\
\begin{align*}
    m &= 266, \ a = 245 \\
    666 &= 245(2) + 176 \\
    245 &= 176(1) + 69 \\
    176 &= 69(2) + 38 \\
    69 &= 38(1) + 31 \\
    38 &= 31(1) + 7 \\
    31 &= 7(4) + 3 \\
    7 &= 3(2) + 1 \\
    3 &= 1(3) + 0
\end{align*}
$421 = m - a$ \\
$69 = 3a - m$ \\
Thus, $x \equiv 473$ is the inverse modulo $666$. We multiply the inverse to both sides
and that gives us $x \equiv 87$.
\\

\textbf{(ii)} $245x \equiv 656 \text{ mod } 666$. \\

We can use the euclidean algorithm to find gcd(245,666) as above.\\
Using the calculations we did above; we multiply the inverse to both sides, which 
gives us $x \equiv 598 \text{ mod } 666$
\\

%----------------------------------------------------------------------------------------
%	PROBLEM 16
%----------------------------------------------------------------------------------------
\textbf{16.} Identify the invertible elements of $\mathbb{Z}_{18}$ and write down their
inverses. \\

\begin{tabular}{|l|l|l|l|l|l|l|l|l|l|l|l|l|l|l|l|l|l|l|}
\hline
$\mathbb{Z}_{18}$ & 0 & 1  & 2  & 3  & 4  & 5  & 6  & 7  & 8  & 9 & 10 & 11 & 12 & 13 & 14 & 15 & 16 & 17 \\ \hline
0                 & 0 & 0  & 0  & 0  & 0  & 0  & 0  & 0  & 0  & 0 & 0  & 0  & 0  & 0  & 0  & 0  & 0  & 0  \\ \hline
1                 & 0 & 1  & 2  & 3  & 4  & 5  & 6  & 7  & 8  & 9 & 10 & 11 & 12 & 13 & 14 & 15 & 16 & 17 \\ \hline
2                 & 0 & 2  & 4  & 6  & 8  & 10 & 12 & 14 & 16 & 0 & 2  & 4  & 6  & 8  & 10 & 12 & 14 & 16 \\ \hline
3                 & 0 & 3  & 6  & 9  & 12 & 15 & 0  & 3  & 6  & 9 & 12 & 15 & 0  & 3  & 6  & 9  & 12 & 15 \\ \hline
4                 & 0 & 4  & 8  & 12 & 16 & 2  & 6  & 10 & 14 & 0 & 4  & 8  & 12 & 16 & 2  & 6  & 10 & 14 \\ \hline
5                 & 0 & 5  & 10 & 15 & 2  & 7  & 12 & 17 & 4  & 9 & 14 & 1  & 6  & 11 & 16 & 3  & 8  & 13 \\ \hline
6                 & 0 & 6  & 12 & 0  & 6  & 12 & 0  & 6  & 12 & 0 & 6  & 12 & 0  & 6  & 12 & 0  & 6  & 12 \\ \hline
7                 & 0 & 7  & 14 & 3  & 10 & 1  & 6  & 13 & 2  & 9 & 16 & 5  & 12 & 1  & 8  & 15 & 4  & 11 \\ \hline
8                 & 0 & 8  & 16 & 6  & 14 & 4  & 12 & 2  & 10 & 0 & 8  & 16 & 6  & 14 & 4  & 12 & 2  & 10 \\ \hline
9                 & 0 & 9  & 0  & 9  & 0  & 9  & 0  & 9  & 0  & 9 & 0  & 9  & 0  & 9  & 0  & 9  & 0  & 9  \\ \hline
10                & 0 & 10 & 2  & 12 & 4  & 14 & 6  & 16 & 8  & 0 & 10 & 2  & 12 & 4  & 14 & 6  & 16 & 8  \\ \hline
11                & 0 & 11 & 4  & 15 & 8  & 1  & 12 & 5  & 16 & 9 & 2  & 13 & 6  & 1  & 10 & 3  & 14 & 7  \\ \hline
12                & 0 & 12 & 6  & 0  & 12 & 6  & 0  & 12 & 6  & 0 & 12 & 6  & 0  & 12 & 6  & 0  & 12 & 6  \\ \hline
13                & 0 & 13 & 8  & 3  & 16 & 11 & 6  & 1  & 14 & 9 & 4  & 17 & 12 & 7  & 2  & 15 & 10 & 5  \\ \hline
14                & 0 & 14 & 10 & 6  & 2  & 16 & 12 & 8  & 4  & 0 & 14 & 10 & 6  & 2  & 16 & 12 & 8  & 4  \\ \hline
15                & 0 & 15 & 12 & 9  & 6  & 3  & 0  & 15 & 23 & 9 & 6  & 3  & 0  & 15 & 12 & 9  & 6  & 3  \\ \hline
16                & 0 & 16 & 14 & 12 & 10 & 8  & 6  & 4  & 2  & 0 & 16 & 14 & 12 & 10 & 8  & 6  & 4  & 2  \\ \hline
17                & 0 & 17 & 16 & 15 & 14 & 13 & 12 & 11 & 10 & 9 & 8  & 7  & 6  & 5  & 4  & 3  & 2  & 1  \\ \hline
\end{tabular}
\\

We can see now that the invertible elements modulo 18 are 1, 5, 7, 11, 13, 17. \\
Their inverses are, in order, 1, 11, 13, 5, 7, 17, modulo 18.

\end{document}
