\documentclass[paper=letter, fontsize=11pt]{scrartcl} % Letter paper and 11pt font size

\usepackage{amstext, amsmath, amssymb}
\usepackage[T1]{fontenc} % Use 8-bit encoding that has 256 glyphs
\usepackage[english]{babel} % English language/hyphenation
\usepackage{amsmath,amsfonts,amsthm} % Math packages

\usepackage{fancyhdr} % Custom headers and footers
\pagestyle{fancyplain} % Makes all pages in the document conform to the custom headers and footers
\fancyhead{} % No page header - if you want one, create it in the same way as the footers below
\fancyfoot[L]{} % Empty left footer
\fancyfoot[C]{} % Empty center footer
\fancyfoot[R]{\thepage} % Page numbering for right footer
\renewcommand{\headrulewidth}{0pt} % Remove header underlines
\renewcommand{\footrulewidth}{0pt} % Remove footer underlines
\setlength{\headheight}{13.6pt} % Customize the height of the header

%----------------------------------------------------------------------------------------
%	TITLE SECTION
%----------------------------------------------------------------------------------------

\newcommand{\horrule}[1]{\rule{\linewidth}{#1}} % Create horizontal rule command with 1 argument of height

\title{	
\normalfont \normalsize 
\textsc{San Francisco State University} \\ [25pt]
\horrule{0.5pt} \\[0.4cm] % Thin top horizontal rule
\huge MATH 301 Assignment 1  \\ % The assignment title
\horrule{2pt} \\[0.5cm] % Thick bottom horizontal rule
}

\author{Omar Sandoval}

\date{\normalsize\today}

\begin{document}

\maketitle

%----------------------------------------------------------------------------------------
%	PROBLEM 1
%----------------------------------------------------------------------------------------

\textbf{1.} By using truth tables prove that, for all statements $P$ and $Q$, the statement (\textit{P} $\Rightarrow$ \textit{Q}) and it's contrapositive $(\neg Q \Rightarrow \neg P)$ are equivalent. In Example 1.2.3 identify which statement is the contrapositive of statement (i) $(f(a)=0 \Rightarrow a >0)$. Find another pair of statements in that list which are contrapositives of each other.
\\\\
\begin{tabular} {| l | c | r | l | c | r |}
	\hline
	P & Q & P $\Rightarrow$ Q & $\neg$P & $\neg$Q & $\neg$Q $\Rightarrow$ $\neg$P \\ \hline
	T & T & T & F & F & T \\ \hline
	T & F & F & F & T & F \\ \hline
	F & T & T & T & F & T \\ \hline
	F & F & T & T & T & T \\ \hline
\end{tabular}
\\\\\\
The converse of (i) $(f(a)=0 \Rightarrow a > 0)$ is $(a >0 \Rightarrow f(a)=0)$.\\
The contrapositive is (vii) For real numbers \textit{a}, if \textit{a} is non-positive then $f(a)\not= 0$. 
\\\\
The statement (iii) If $f(a) = 0$ then $a$ is non-positive, has a contrapositve, (vi) If $a$ is positive then $f(a) \not= 0$.
\\

%----------------------------------------------------------------------------------------
%	PROBLEM 2
%----------------------------------------------------------------------------------------

\textbf{2.} By using truth tables prove that, for all statements $P$ and $Q$ , the three statements (i)  $P \implies Q$ (ii) $(P$ or $Q )$ $\leftrightarrow Q$ and (iii) $( P$ and $Q )$ $\leftrightarrow P$ are equivalent.

\begin{tabular} {| l | c | r | l | c | r | l |}
	\hline
	P & Q & P or Q & P and Q & P $\implies$ Q & (P or Q) $\leftrightarrow$ Q & (P and Q) $\leftrightarrow$ Q\\ \hline
	T & T & T & T & T & T & T\\ \hline
	T & F & T & F & F & F & F\\ \hline
	F & T & T & F & T & T & T\\ \hline
	F & F & T & F & T & T & T\\ \hline
\end{tabular}
\\\\
%----------------------------------------------------------------------------------------
%	PROBLEM 4
%----------------------------------------------------------------------------------------

\textbf{4.} Prove the following statements concerning positive integers a, b and c.
\\\\
(i) ( $a$ divides $b$ ) and ( $a$ divides $c$ ) $\implies$ $a$ divides ( $b$ + $c$ ).

$a$ divides $b$ means, for some integer $q$, $b=aq$.\\
$a$ divides $c$ means, for some integer $p$, $c=ap$.\\
So, $b + c = aq + ap = a(q + p) = ak$, where $k = q + p$ for some integer.\\
Therefore, $a$ divides $b + c$.\\

(ii) ( $a$ divides $b$ ) or ( $a$ divides $c$) $\implies$ $a$ divides $bc$.\\

Proof by Cases \\
Case 1: $a | b$ means $b = aq$ for some integer $q$.\\
So, $bc = (aq)c = a(qc) = ar$, where $r = qc$ is an integer.\\
Therefore, $a | bc$.\\

Case 2: $a | c$ means $c = ap$ for some integer $p$.\\
So, $bc = b(ap) = a(bp) = as$, where $s = bp$ is an integer.\\
Therefore, $a | bc$.\\

Clearly, $a | bc$.\\\\

%----------------------------------------------------------------------------------------
%	PROBLEM 5
%----------------------------------------------------------------------------------------

\textbf{5.} Which of the following conditions are necessary for the positive integer $n$ to be divisible by 6 (proofs not necessary)?\\
(i) 3 divides n.\\
(ii) 9 divides n.\\
(iii) 12 divides n.\\
(iv) n = 12.\\
(v) 6 divides  $n^2$ .\\
(vi) 2 divides n and 3 divides n.\\
(vii) 2 divides n or 3 divides n.\\

Which of these conditions are sufficient?\\

When 6 divides $n$, for some integer $q$, $n=6q$.\\
The conditions (i), (v), (vi) and (vii) are \textit{necessary} for positive integer $n$ to be divisible by 6.\\

The conditions (iii), (iv), (v), and (vi) are \textit{sufficient} for positive integer $n$ to be divisible by 6.\\

%----------------------------------------------------------------------------------------
%	PROBLEM 6
%----------------------------------------------------------------------------------------
\textbf{6.} User the properties of addition and multiplication of real numbers given in Properties 2.3.1 to deduce that, for all real numbers $a$ and $b$.\\

(i) $a \times 0 = 0 = 0 \times a$.

Proof\\
$0 + 0 = 0$	$\hspace{1 in}$ Additive Identity.\\
$(0 \times a) + (0 \times a) = (0 \times a)$  Distributive property.\\
$0 \times a = 0$ $\hspace{1 in}$ Additive inverse.\\
$a \times 0 = 0$ $\hspace{1 in}$ Commutative property.\\
Therefore, $ a \times 0 = 0 = 0 \times a$.\\

(ii) $(-a)b = -ab = a(-b)$.\\

We first need to prove that $-a = (-1)a$.
\begin{align*}
	a + (-1)a &= 1a + (-1)a &\text{Multiplicative iden.}\\
	&= (1 + (-1))a &\text{Distributive prop.}\\
	&= 0a \\
	&= 0
\end{align*}
So, $a + (-1)a = 0 \rightarrow -a = (-1)a$.\\
\begin{align*}
	-ab &= (-1)ab \\
	&= ((-1)a)b &\text{Associative}\\
	&= (-a)b\\
	-ab &= (-1)ab\\
	&= ((-1)b)a &\text{Associative}\\
	&=(-b)a\\
	&= a(-b) &\text{Commutative}
\end{align*}
Therefore, $(-a)b  = -ab = a(-b)$.\\

(iii) $(-a)(-b) = ab$.\\
\begin{align*}
	-a = (-1)a \\
	-b = (-1)b \\
	a &= -(-a) &\text{Additive Iden.}\\
	ab &= -(-a)b \\
	&= -(-ab) \\
	&= -((-a)b) \\
	&=-(b(-a)) \\
	&=(-b)(-a) &\text{Commutative prop.}
\end{align*}
Therefore, $(-a)(-b) = ab$.


\end{document}