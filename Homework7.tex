\documentclass[paper=letter, fontsize=11pt]{scrartcl} % Letter paper and 11pt font size

\usepackage{amstext, amsmath, amssymb}
\usepackage[T1]{fontenc} % Use 8-bit encoding that has 256 glyphs
\usepackage[english]{babel} % English language/hyphenation
\usepackage{amsmath,amsfonts,amsthm} % Math packages

\usepackage{fancyhdr} % Custom headers and footers
\pagestyle{fancyplain} % Makes all pages in the document conform to the custom headers and footers
\fancyhead{} % No page header
\fancyfoot[L]{} % Empty left footer
\fancyfoot[C]{} % Empty center footer
\fancyfoot[R]{\thepage} % Page numbering for right footer
\renewcommand{\headrulewidth}{0pt} % Remove header underlines
\renewcommand{\footrulewidth}{0pt} % Remove footer underlines
\setlength{\headheight}{13.6pt} % Customize the height of the header
\setlength\parindent{0pt} % Remove all indentation from paragraps.

%----------------------------------------------------------------------------------------
%	TITLE SECTION
%----------------------------------------------------------------------------------------

\newcommand{\horrule}[1]{\rule{\linewidth}{#1}} % Create horizontal rule command with 1 argument of height

\title{	
\normalfont \normalsize 
\textsc{San Francisco State University} \\ [25pt]
\horrule{0.5pt} \\[0.4cm] % Thin top horizontal rule
\huge MATH 301 Assignment 7  \\ % The assignment title
\horrule{2pt} \\[0.5cm] % Thick bottom horizontal rule
}

\author{Omar Sandoval}

\date{\normalsize\today}

\begin{document}

\maketitle

%----------------------------------------------------------------------------------------
%	PROBLEM #19
%----------------------------------------------------------------------------------------
\textbf{19.} Prove Leibniz's rule for higher order derivatives of products,
\begin{align*}
    \dfrac{d^n(uv)}{dx^n} = \sum_{i=0}^{n} \dbinom{n}{i} \dfrac{d^iu}{dx^i} \dfrac{d^{n-
    i}v}{dx^{n-i}} \text{ for } n \in \mathbb{Z}^{+}
\end{align*}
by induction on $n$. \\
\\

Base Case: When $i = 0$. \\

$\dfrac{d(uv)}{dx^k} = u \dfrac{dv}{dx} \dfrac{du}{dx}v$

Inductive Step: $P(k) \Rightarrow P(k+1)$ \\ 
Need to show: $P(k+1): \dfrac{d^{k+1}(uv)}{dx^{k+1}} = \sum\limits_{i=0}^{k+1} \dbinom{k+1}{i}
\dfrac{d}{dx}\dfrac{d^iu}{dx^i}\dfrac{d^{k+1-i}v}{dx^{k+1-i}}$

We can see that; $\dfrac{d^{k+1}(uv)}{dx^{k+1}} = \dfrac{d}{dx} \dfrac{d^k(uv)}{dx^k} =$
$\sum\limits_{i=0}^{k}\dbinom{k}{i} \dfrac{d^{i+1}u}{dx^{i+1}}\dfrac{d^{k-i}v}{dx^{k-i}}$
$+ \sum\limits_{i=0}^{k}\dbinom{k}{i} \dfrac{d^iu}{dx^i}\dfrac{d^{k+1-i}v}{dx^{k+1-i}}$.

By replacing $i + 1$ with $i$ and $i$ with $i-1$, and extending our summation to $k+1$ we can get the following; \\


$\sum\limits_{i=0}^{k}\dbinom{k}{i} \dfrac{d^{i+1}u}{dx^{i+1}}\dfrac{d^{k-i}v}{dx^{k-i}}$
$ = \sum\limits_{i=0}^{k+1}\dbinom{k}{i-1} \dfrac{d^{i}u}{dx^{i}}\dfrac{d^{k+1-i}v}{dx^{k+1-i}}$ \\

So,
\begin{align*}
    \dfrac{d^{k+1}(uv)}{dx^{k+1}} &= \sum_{i=1}^{k+1}\dbinom{k}{i-1} \dfrac{d^{i}u}
    {dx^{i}}\dfrac{d^{k+1-i}v}{dx^{k+1-i}} + \sum\limits_{i=0}^{k+1}\dbinom{k}{i-1}
    \dfrac{d^{i}u}{dx^{i}}\dfrac{d^{k+1-i}v}{dx^{k+1-i}} \\
    &= \sum_{i=1}^{k}(\dbinom{k}{i-1}+\dbinom{k}{i})\dfrac{d^iu}{dx^i}\dfrac{d^{k+1-i}v}
    {dx^{k+1-i}} + \dbinom{k}{k}\dfrac{d^{k+1}u}{dx^{k+1}} + \dbinom{k}{0}\dfrac{d^{k+1}v}{dx^{k+1}} \\
    &= \sum_{i=0}^{k+1}\dbinom{k+1}{i}\dfrac{d^iu}{dx^i}\dfrac{d^{k+1-i}v}{dx^{k+1-i}}
\end{align*}

Since $\dbinom{k}{i-1} + \dbinom{k+1}{i}$ and $\dbinom{k}{k} = \dbinom{k+1}{k+1}$

%----------------------------------------------------------------------------------------
%	PROBLEM #23
%----------------------------------------------------------------------------------------
\textbf{23.} Prove that there does not exist a rational number whose suqare is 10. \\
\\

Proof by Contradiction \\
Suppose for contradiction that there does exist a rational number, $x$, whose square is 
divisible by 10. Then, there exists integers $p$ and $q$ with no common divisors, so, 
$x = \dfrac{p}{q}$ and $x^2 = 10$ \\

So, $\dfrac{p^2}{q^2} = 10$, and $p^2 = 10q^2$ which tells us that $p$ is divisible by $10$
. That is, there is an integer $a$ so that $p = 10a$. Hence, $10q^2 = p^2 = (10a)^2 = 
100a^2$. So, $q^2 = 10a^2$ which means that $q$ is also divisible by 10, which is a 
contradiction to our assumption. \\
\\

%----------------------------------------------------------------------------------------
%	PROBLEM #25
%----------------------------------------------------------------------------------------
\textbf{25.} Find the rational number equal to the recurring infinite decimal $2.10012\dot{0}9\dot{7}$. \\

Let $x = 2.10012\dot{0}9\dot{7}$. Then $1000000x = 2100120.97\dot{0}9\dot{7}$ from which it follows that $999999x = 2100120.97 - 2.10012 = 2100118.87 = 210011887/100$. \\
It follows
that $x = 210011887/99999900$

\end{document}
