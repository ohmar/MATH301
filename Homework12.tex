\documentclass[paper=letter, fontsize=11pt]{scrartcl} % Letter paper and 11pt font size

\usepackage{amstext, amsmath, amssymb}
\usepackage[T1]{fontenc} % Use 8-bit encoding that has 256 glyphs
\usepackage[english]{babel} % English language/hyphenation
\usepackage{amsmath,amsfonts,amsthm} % Math packages

\usepackage{fancyhdr} % Custom headers and footers
\pagestyle{fancyplain} % Makes all pages in the document conform to the custom headers and footers
\fancyhead{} % No page header
\fancyfoot[L]{} % Empty left footer
\fancyfoot[C]{} % Empty center footer
\fancyfoot[R]{\thepage} % Page numbering for right footer
\renewcommand{\headrulewidth}{0pt} % Remove header underlines
\renewcommand{\footrulewidth}{0pt} % Remove footer underlines
\setlength{\headheight}{13.6pt} % Customize the height of the header
\setlength\parindent{0pt} % Remove all indentation from paragraps.

%----------------------------------------------------------------------------------------
%	TITLE SECTION
%----------------------------------------------------------------------------------------

\newcommand{\horrule}[1]{\rule{\linewidth}{#1}} % Create horizontal rule command with 1 argument of height

\title{	
\normalfont \normalsize 
\textsc{San Francisco State University} \\ [25pt]
\horrule{0.5pt} \\[0.4cm] % Thin top horizontal rule
\huge MATH 301 Assignment 4  \\ % The assignment title
\horrule{2pt} \\[0.5cm] % Thick bottom horizontal rule
}

\author{Omar Sandoval}

\date{\normalsize\today}

\begin{document}

\maketitle

%p. 296: 12, 13, 16 

%----------------------------------------------------------------------------------------
%	PROBLEM #12
%----------------------------------------------------------------------------------------
\textbf{12.} Prove that if the polynomial equation $a_n x^n + \dots + a_1 x + a_0 = 0$
(with integer coefficients) has a rational solution $x = r / s$ (in its lowest terms)
then $r$ divides $a_0$ and $s$ divides $a_n$. \\
Deduce that every rational solution of $x^n + a_{n-1} x^{n-1} + \dots + a_1 x + a_0 = 0$
is an integer. \\

Hence, show that the equation $x^3 - 8x + a = 0$ has no rational solutions if
$a \equiv \pm 1 \text{ mod } 5.$
\\

%----------------------------------------------------------------------------------------
%	PROBLEM #13
%----------------------------------------------------------------------------------------
\textbf{13.} Recall that, given a prime $p$, the order modulo $p$ of an integer
$a \not\equiv 0 \text{ mod } p$ is the least positive integer $n$ such that 
$a^n \equiv 1 \text{ mod } p$. Find the orders modulo $p$ of all the non-zero elements
of $R_p$ for $p = 5$ and $p = 11$. \\

%----------------------------------------------------------------------------------------
%	PROBLEM #16
%----------------------------------------------------------------------------------------
\textbf{16.} Find the remainders when \\
\textbf{(i)} $29!$ is divided by 31, \\
\textbf{(ii)} $18!$ is divided by 23, \\
\textbf{(iii)} $18!$ is divided by 437.
\\

\end{document}
